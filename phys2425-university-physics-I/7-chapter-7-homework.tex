% Copyright (C) Dean Gadberry - All Rights Reserved
% Unauthorized copying of this file, via any medium is strictly prohibited
% Proprietary and confidential
% Written by Dean Gadberry <dean@deangadberry.com>, 2023

\documentclass[12pt,a4paper,english]{article}
\usepackage{helvet}         % set font to helvetica
\renewcommand{\familydefault}{\sfdefault}
\usepackage{sectsty}        % allow redefinition of section command formatting
    \allsectionsfont{\normalsize}
\usepackage[centering,noheadfoot,margin=1in]{geometry}
\usepackage{lipsum}
\setlength{\parindent}{0em}
\usepackage{parskip}
\usepackage{hanging}
\usepackage{hyperref}
\hypersetup{colorlinks=true, linkcolor=cyan, urlcolor=cyan}
\usepackage{amsmath}
\usepackage{esvect}
\usepackage{currfile}
\setcounter{secnumdepth}{0}

\newcommand{\degree}[1]{${#1}^\circ$}

% https://tex.stackexchange.com/questions/30720/footnote-without-a-marker
\newcommand\blindfootnote[1]{
  \begingroup
  \renewcommand\thefootnote{}\footnote{#1}
  \addtocounter{footnote}{-1}
  \endgroup
}
\AtEndDocument{\blindfootnote{\textrm{This document proudly made using \LaTeX{}}}}

\begin{document}
\begin{flushright}
  Dean Gadberry

  NCTC PHYS 2425

  \today
\end{flushright}
\begin{center}
  {\large Chapter 7 Homework}
\end{center}
\begin{flushleft}

  \section*{Questions}
  \hrule
  \subsection{11. If the speed of a particle doubles, by what factor does its kinetic energy increase?}
  \begin{align*}
    k=\frac{1}{2}mv^2       \rightarrow
    k=\frac{1}{2}m(2v)^2    \rightarrow
    k=\frac{1}{2}m(2)^2(v)^2\rightarrow
    k=\frac{1}{2}m(4)(v)^2
  \end{align*}
  Because kinetic energy is exponentially related to velocity, the kinetic energy of a particle, for which speed doubles, would quadruple. There was an interesting example of this phenomenon demonstrated in class, in which the distance to stop a car was compared for various speeds.
  \section*{Misconception Questions}
  \hrule
  \subsection{2. You are carrying a 10-kg bag and moving at constant speed. In which case will you do the most work on the bag?}
  (d) Climb up a 5-m-tall slope.
  \subsection{5. If you push twice as hard against a stationary brick wall, the amount of work you do}
  (d) is zero.
  \subsection{7. A delivery man carrying a package walks up the stairs to the second floor at constant speed (A), and along the hall at a constant speed (B). He accelerates to a run and then moves at a greater constant speed along the hall. During what portions of his motion is the delivery man doing work on the package? (ignore friction)}
  (b) C only.
  \section*{Problems}
  \hrule
  \subsection{1. How much work is done by gravitational force when a 280-kg pile driver falls 3.80m?}
Earth's mass and radius are $m_E=5.98\times10^{24}kg$ and $r_E=6.38\times10^6m$
  \begin{align*}
    W&=F_{||}d
     &
    F_G&=G\frac{m_1m_2}{r^2} 
    \\
     &
     &
    F_G&=G\frac{m_Em_P}{(r_E+3.80)^2} 
    \\
     &
     &
    F_G&=(6.67\times10^{-11}N\frac{m^2}{kg^2})\frac{(5.98\times10^{24}kg)(280kg)}{(6.38\times10^6m+3.80m)^2} 
    \\
    W&=(2743N)(3.80m)
     &
    F_G&=2743.74N
    \\
    W&=10426J
    \\
    W&=1.0\times10^4J
  \end{align*}
  \subsection{3. A 55.0-kg firefighter climbs a flight of stairs 28.0 m high at constant speed. How much work does she do?}
  \begin{align*}
    W&=F_{||}d
     &
    \Sigma F_y=F_H-mg
     &=0
     \\
     &
     &
  F_H&=mg
     \\
  W&=mgd
  \\
  W&=(55.0kg)(9.8m/s^2)(28.0m)
  \\
  W&=15092J
  \\
  W&=1.5\times10^{4}J
  \end{align*}
  \subsection{18. What is the dot product of $\vv{A}=2.0x^2\hat{i}-4.0x\hat{j}+5.0\hat{k}$ and $\vv{B}=11.0\hat{i}+2.5x\hat{j}$?}
  \begin{align*}
    \vv{A}\cdot\vv{B}&=A_xB_x+A_yB_y+A_zB_z
    \\
                     &=(2.0x^2)(11.0)+(-4.0x)(2.5x)+(5.0)(0)
                     \\
                     &=22x^2-10x^2+0
                     \\
                     &=12x^2
  \end{align*}
  \subsection{24. A constant force $\vv{F}=(2.0\hat{i}+4.0\hat{j})N$ acts on an object as it moves along a straight-line path. If the object's displacement is $\vv{d}=(1.0\hat{i}+5.0\hat{j})m$, calculate the work done by $\vv{F}$ using these alternate ways of writing the dot product:}
  \begin{align*}
$(a)$W&=Fdcos\Theta
      &
$(b)$W&=F_xd_x+F_yd_y
      \\
      &=\Bigr(\sqrt{(2.0)^2+(4.0)^2}N\Bigr)(5.0m)
      &
      &=(2.0N)(1.0m)+(4.0N)(5.0m)
      \\
      W&=22J
      &
      W&=22J
  \end{align*}
  \subsection{29. Let $\vv{V}=20.0\hat{i}+26.0\hat{j}-14.0\hat{k}$. What angles does this vector make with the $x$, $y$, and $z$ axes?}
  \begin{align*}
    R=\sqrt{(20.0)^2+(26.0)^2+(-14.0)^2}&=35.7
    \\
    \Theta_x=arccos\frac{20.0}{35.7}&=
    55.9^\circ
    \\
    \Theta_y=arccos\frac{26.0}{35.7}&=
    43.2^\circ
    \\
    \Theta_z=arccos\frac{-14.0}{35.7}&=
    113^\circ
    \\
  \end{align*}
  \subsection{37. A spring has $k=65N/m$. Draw a graph like that in Fig. 7-11 and use it to determine the work needed to stretch the spring from $x=3.02cm$ to $x=7.5cm$, where $x=0$ refers to the spring's unstretched length.}
  \begin{align*}
    W&=\frac{1}{2}kx^2
    \\
    W_1&=\frac{1}{2}(65N/m)(0.0302m)
       &
       &
       &
    W_2&=\frac{1}{2}(65N/m)(0.075m)
    \\
    W_1&=0.9815J
       &
       &
       &
    W_2&=2.4375J
    \\
       &
       &
    W_2-W_1&=1.5J
  \end{align*}
  \subsection{39. The net force exerted on a particle acts in the positive x direction. Its magnitude increases linearly from zero at $x=0$, to $380N$ at $x=3.0m$. It remains constant at $380N$ from $x=3.0m$ to $x=7.0m$, and then decreasees linearly to zero at $x=12.0m$. Determine the work done to move the particle from $x=0$ to $x=12.0m$ graphically, by determining the area under the $F_x$ versus $x$ graph.}
  \begin{align*}
    W&=\frac{1}{2}(380N)(3.0m)+(380N)(4.0m)+\frac{1}{2}(380N)(5)
    \\
    W&=3040J
  \end{align*}
  \subsection{56. How much work is required to stop an electron $(m=9.11\times10^{-31}kg)$ which is moving with a speed of $1.10\times10^6m/s$?}
  \begin{align*}
    W&=\frac{1}{2}mv_1^2-\frac{1}{2}mv_2^2
    %\rightarrow 
    %W=\frac{1}{2}(mv_1^2-mv_2^2)
    \\
  W&=\frac{1}{2}(9.11\times10^{-31}kg)(1.10\times10^6m/s)^2-0
  \\
  W&=5.51\times10^{-19}J
  \end{align*}
  \subsection{62. At an accident scene on a level road, investigators measure a car's skid mark to be 78m long. It was a rainy day and the coefficient of friction was estimated to be 0.30.}
  \subsubsection{(a) Use these data to determine the speed of the car when the driver slammed on (and locked) the brakes.}
  \begin{align*}
    \Sigma F_x=0-F_{fr}
    &=ma
    &
    \Sigma F_y=F_N-mg&=0
    \\
    -\mu F_N&=ma
            &
    F_N&=mg
    \\
    -\mu mg&=ma
           &
           &&v_f^2&=v_0^2+2a\Delta x
    \\
    -\mu g&=a
    %(0.30)(9.8m/s^2)=2.94m/s^2
          &
          &&v_f^2&=v_0^2+2(^-\mu g)\Delta x
    \\
          &&&&v_0&=\sqrt{2\mu g\Delta x}
          \\
          &&&&v_0&=\sqrt{2(0.30)(9.8m/s^2)(78m)}
          \\
          &&&&v_0&=21.4m/s
  \end{align*}
  \subsubsection{(b) Why does the car's mass not matter?}
  The mass of the car is included in each of the sumation of force equations. When these equations are substitued into one another, the mass is algebraically irrelevant.
  \subsubsection{(c) What is wrong with a car that skids (see page 131)?}
  The car which skids is losing traction. Maybe the vehicle-owner needs new tires on his old Corolla! And the car may have faulty Anti-lock-braking.
  \subsection{74. Spiderman uses his spider webs to save a runaway train moving about $60km/h$. His web stretches a few city blocks ($500 m$) before the $10^4kg$ train comes to a stop. Assuming the web acts like a spring, estimate the effective spring constant.}
  \begin{align*}
    F_W&=ma
       &
    k&=\frac{F}{x}
     &
     v_f^2&=v_0^2+2a\Delta x
    \\
          &&&&a&=\frac{v^2_0}{2\Delta x}
          \\
          &&&&a&=\frac{-(16.\overline{6}m/s)^2}{2(500m)}
          \\
     F_W&=(10^4kg)(^-0.2\overline{7}m/s^2)
        &&&\leftarrow a&=-0.2\overline{7}m/s^2
          \\
     F_W&=-2700N\rightarrow
        &
     k&=\frac{^-0.2\overline{7}\times10^4N}{-500m}
     \\
      &&
     k&=5.\overline{5}N/m
  \end{align*}
\end{flushleft}
\end{document}
