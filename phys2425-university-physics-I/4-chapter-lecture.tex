% Copyright (C) Dean Gadberry - All Rights Reserved
% Unauthorized copying of this file, via any medium is strictly prohibd
% Proprietary and confidential
% Written by Dean Gadberry <dean@deangadberry.com>, 2023

\documentclass[12pt,a4paper,english]{article}
\usepackage{helvet}         % set font to helvetica
\renewcommand{\familydefault}{\sfdefault}
\usepackage{sectsty}        % allow redefinition of section command formatting
    \allsectionsfont{\normalsize}
\usepackage[centering,noheadfoot,margin=1in]{geometry}
\usepackage{lipsum}
\setlength{\parindent}{0em}
\usepackage{parskip}
\usepackage{hanging}
\usepackage{hyperref}
\hypersetup{colorlinks=true, linkcolor=cyan, urlcolor=cyan}
\usepackage{amsmath}
\usepackage{esvect}
\usepackage{currfile}
\setcounter{secnumdepth}{0}

\newcommand{\degree}[1]{${#1}^\circ$}

% https://tex.stackexchange.com/questions/30720/footnote-without-a-marker
\newcommand\blindfootnote[1]{
  \begingroup
  \renewcommand\thefootnote{}\footnote{#1}
  \addtocounter{footnote}{-1}
  \endgroup
}
\AtEndDocument{\blindfootnote{\textrm{This document proudly made using \LaTeX{}}}}

\begin{document}
\begin{flushright}
  Dean Gadberry

  NCTC PHYS 2425

  \today
\end{flushright}
\begin{center}
  {\large Chapter 4 Lecture}
\end{center}
\begin{flushleft}

  \section*{heading}
  \hrule
  \subsection{subheading}

  Newton's Laws
  \begin{enumerate}
      \item Every object continues in its state of rest, or of uniform velocity in a straight line, as long as no net force acts on it.
      \item The acceleration of an object is directly proportional to the net force acting on it, and is inversely proportional to the object's mass. The direction of the acceleration is in the direction of the net force acting on the object.
      \item Whenever one object exerts force on a second object, the second object exerts an equal force in the opposite direction on the first object.
  \end{enumerate}

  Newtons:
  \[
      acceleration=\frac{Force}{mass}\implies
      a=\frac{F}{m}\implies
      N=\frac{1kg\times m}{s^2}
  \]
\end{flushleft}
\end{document}
