% Copyright (C) Dean Gadberry - All Rights Reserved
% Unauthorized copying of this file, via any medium is strictly prohibited
% Proprietary and confidential
% Written by Dean Gadberry <dean@deangadberry.com>, 2023

\documentclass[12pt,a4paper,english]{article}
\usepackage{helvet}         % set font to helvetica
\renewcommand{\familydefault}{\sfdefault}
\usepackage{sectsty}        % allow redefinition of section command formatting
    \allsectionsfont{\normalsize}
\usepackage[centering,noheadfoot,margin=1in]{geometry}
\usepackage{lipsum}
\setlength{\parindent}{0em}
\usepackage{parskip}
\usepackage{hanging}
\usepackage{hyperref}
\hypersetup{colorlinks=true, linkcolor=cyan, urlcolor=cyan}
\usepackage{amsmath}
\usepackage{esvect}
\usepackage{currfile}
\setcounter{secnumdepth}{0}

\newcommand{\degree}[1]{${#1}^\circ$}

% https://tex.stackexchange.com/questions/30720/footnote-without-a-marker
\newcommand\blindfootnote[1]{
  \begingroup
  \renewcommand\thefootnote{}\footnote{#1}
  \addtocounter{footnote}{-1}
  \endgroup
}
\AtEndDocument{\blindfootnote{\textrm{This document proudly made using \LaTeX{}}}}

\begin{document}
\begin{flushright}
  Dean Gadberry

  NCTC PHYS 2425

  \today
\end{flushright}
\begin{center}
  {\large Chapter 6 Homework}
\end{center}
\begin{flushleft}

  \section*{Questions}
  \hrule
  \subsection{3. Will an object weigh more at the equator or at the poles? What two effects are at work? Do they oppose each other?}
  An object will weigh more at the poles than at the equator. The two effects are gravity and centripetal force. Centripetal force tangentially opposes gravity.
  \subsection{10. Would it require less speed to launch a satellite (a) toward the east or (b) toward the west? Consider the Earth's rotation direction, and explain your choice.}
  The Sun rises in the East and sets in the West. This zetetic phenomenon is a feature of the Earth spinning in the Easterly direction. Were the Earth some play-thing, objects released as it spins would tend to continue East. Thus, it would require less speed to launch a satellite (a) toward the east.
  \section*{Misconception Questions}
  \hrule
  \subsection{2. In the international Space Station which orbits Earth, astronauts experience apparent weightlessness because}
  (c) the astronauts and the station are in free fall towards the center of the Earth.
  \subsection{3. Which pulls harder gravitationally, the Earth on the Moon, or the Moon on the Earth? Which accelerates more?}
  (f) Both the same; the Moon.
  \subsection{6. As you travel away from Earth's surface,}
  (a) your weight decreases and your mass remains the same. 
  \section*{Problems}
  \hrule
  \subsection{1. Calculate the force of Earth's gravity on a spacecraft 2.00 Earth radii above the Earth's surface if its mass is 1650 kg.}
  \subsection{4. A hypothetical planet has a radius 2.5 times that of Earth, but has the same mass. What is the acceleration due to gravity hear its surface?}
  \subsection{5. A hypothetical planet has a mass 2.80 times that of Earth, but has the same radius. What is g near its surface?}
  \subsection{31. What will a spring scale read for the weight of a 58.0-kg woman in an elevator that moves...}
  \subsubsection{(a) upward with constant speed 4.4m/s}
  \subsubsection{(b) downward with constant speed 4.4m/s}
  \subsubsection{(c) with an upward acceleration 0.18g}
  \subsubsection{(d) with a downward acceleration 0.18g}
  \subsubsection{(e) in free fall?}
  \subsection{32. Astronomers using the Hubble Space Telescope deduced the presence of an extremely massive core in the distant galaxy M87, so dense that it could be a black hole (from which no light escapes). They did this by measuring the speed of gas clouds orbiting the core to be 780km/s at a distance of 60 light-years ($=5.7\times10^{17}m$) from the core. Deduce the mass of the core, and compare it to the mass of our Sun.}
\end{flushleft}
\end{document}
