% Copyright (C) Dean Gadberry - All Rights Reserved
% Unauthorized copying of this file, via any medium is strictly prohibited
% Proprietary and confidential
% Written by Dean Gadberry <dean@deangadberry.com>, 2023

\documentclass[12pt,a4paper,english]{article}
\usepackage{helvet}         % set font to helvetica
\renewcommand{\familydefault}{\sfdefault}
\usepackage{sectsty}        % allow redefinition of section command formatting
    \allsectionsfont{\normalsize}
\usepackage[centering,noheadfoot,margin=1in]{geometry}
\usepackage{lipsum}
\setlength{\parindent}{0em}
\usepackage{parskip}
\usepackage{hanging}
\usepackage{hyperref}
\hypersetup{colorlinks=true, linkcolor=cyan, urlcolor=cyan}
\usepackage{amsmath}
\usepackage{esvect}
\usepackage{currfile}
\setcounter{secnumdepth}{0}

\newcommand{\degree}[1]{${#1}^\circ$}

% https://tex.stackexchange.com/questions/30720/footnote-without-a-marker
\newcommand\blindfootnote[1]{
  \begingroup
  \renewcommand\thefootnote{}\footnote{#1}
  \addtocounter{footnote}{-1}
  \endgroup
}
\AtEndDocument{\blindfootnote{\textrm{This document proudly made using \LaTeX{}}}}

\begin{document}
\begin{flushright}
  Dean Gadberry

  NCTC PHYS 2425

  \today
\end{flushright}
\begin{center}
  {\large Chapter 6 Homework}
\end{center}
\begin{flushleft}

  \section*{Questions}
  \hrule
  \subsection{3. Will an object weigh more at the equator or at the poles? What two effects are at work? Do they oppose each other?}
  An object will weigh more at the poles than at the equator. The two effects are gravity and centripetal force. Centripetal force tangentially opposes gravity.
  \subsection{10. Would it require less speed to launch a satellite (a) toward the east or (b) toward the west? Consider the Earth's rotation direction, and explain your choice.}
  The Sun rises in the East and sets in the West. This zetetic phenomenon is a feature of the Earth spinning in the Easterly direction. Were the Earth some play-thing, objects released as it spins would tend to continue East. Thus, it would require less speed to launch a satellite (a) toward the east.
  \section*{Misconception Questions}
  \hrule
  \subsection{2. In the international Space Station which orbits Earth, astronauts experience apparent weightlessness because}
  (c) the astronauts and the station are in free fall towards the center of the Earth.
  \subsection{3. Which pulls harder gravitationally, the Earth on the Moon, or the Moon on the Earth? Which accelerates more?}
  (f) Both the same; the Moon.
  \subsection{6. As you travel away from Earth's surface,}
  (a) your weight decreases and your mass remains the same. 
  \pagebreak
  \section*{Problems}
  \hrule
  \subsection{1. Calculate the force of Earth's gravity on a spacecraft 2.00 Earth radii above the Earth's surface if its mass is 1650 kg. (Remember that Earth's mass and radius are given by the equations $m_E=5.98\times10^{24}kg$ and $r_E=6380km$)}
  \begin{align*}
    F&=G\frac{m_1m_2}{r^2} 
     &\frac{6380km}{1}\biggr(\frac{1000m}{1km}\biggr)&=6.38\times10^{6}m
    \\
    F&=\biggr(6.67\times10^{-11}N\frac{m^2}{kg^2}\biggr)\frac{(5.98\times10^{24}kg)(1650kg)}{\bigr((3)(6.38\times10^6m)\bigr)^2} 
    \\
    F&=3.81\times10^{3}N
    \\
  \end{align*}
  \subsection{4. A hypothetical planet has a radius 2.5 times that of Earth, but has the same mass. What is the acceleration due to gravity near its surface?(Remember that Earth's mass and radius are given by the equations $m_E=5.98\times10^{24}kg$ and $r_E=6380km$)}
  \begin{align*}
    F&=G\frac{m_1m_2}{r^2} 
    \\
    F&=\biggr(6.67\times10^{-11}N\frac{m^2}{kg^2}\biggr)\frac{(5.98\times10^{24}kg)}{\bigr((2.5)(6.38\times10^6m)\bigr)^2} 
    \\
    F&=1.57m/s^2
  \end{align*}
  \subsection{5. A hypothetical planet has a mass 2.80 times that of Earth, but has the same radius. What is g near its surface?(Remember that Earth's mass and radius are given by the equations $m_E=5.98\times10^{24}kg$ and $r_E=6380km$)}
  \begin{align*}
    F&=G\frac{m_1m_2}{r^2} 
    \\
    F&=\biggr(6.67\times10^{-11}N\frac{m^2}{kg^2}\biggr)\frac{(2.80)(5.98\times10^{24}kg)}{(6.38\times10^6m)^2} 
    \\
    F&=27.4m/s^2
  \end{align*}
  \subsection{31. What will a spring scale read for the weight of a 58.0-kg woman in an elevator that moves...}
  \begin{align*}
    w-mg&=ma
    \\
    w&=mg+ma
    \\
    w&=m(g+a)
    \\
    w&=58.0kg(9.8m/s^2+a)
    \\
  \end{align*}
  \subsubsection{(a) upward with constant speed 4.4m/s}
  \begin{align*}
    w&=58.0kg(9.8m/s^2)(1+0)
    \\
    w&=568N
  \end{align*}
  \subsubsection{(b) downward with constant speed 4.4m/s}
  \begin{align*}
    w&=58.0kg(9.8m/s^2)(1-0)
    \\
    w&=568N
  \end{align*}
  \subsubsection{(c) with an upward acceleration 0.18g}
  \begin{align*}
    w&=58.0kg(9.8m/s^2)(1+0.18)
    \\
    w&=58.0kg(9.8m/s^2)(1.18)
    \\
    w&=671N
  \end{align*}
  \subsubsection{(d) with a downward acceleration 0.18g}
  \begin{align*}
    w&=58.0kg(9.8m/s^2)(1-0.18)
    \\
    w&=58.0kg(9.8m/s^2)(0.82)
    \\
    w&=470N
  \end{align*}
  \subsubsection{(e) in free fall?}
  \begin{align*}
    w&=58.0kg(9.8m/s^2)(1-1)
    \\
    w&=0N
  \end{align*}
  \subsection{32. You know your mass is 62kg, but when you stand on a bathroom scale in an elevator, it says your mass is 72kg. What is the acceleration of the elevator, and in which direction?}
  \begin{align*}
    w&=72kg(9.8m/s^2)=705.6N
    \\
    705.6N&=62kg(9.8m/s^2)(1+a)
    \\
    705.6N&=62kg(9.8m/s^2)(1+a)
    \\
    %705.6N&=607.6N(1+a)
    %\\
    \frac{705.6N}{607.6N}&=\frac{607.6N}{607.6N}(1+a)
    \\
    \frac{705.6N}{607.6N}-1&=a
    \\
    0.161&=a $ (as it relates to g)$
    \\
    1.58m/s^2
  \end{align*}
  The elevator is moving up.
\end{flushleft}
\end{document}
