% Copyright (C) Dean Gadberry - All Rights Reserved
% Unauthorized copying of this file, via any medium is strictly prohibd
% Proprietary and confidential
% Written by Dean Gadberry <dean@deangadberry.com>, 2023

\documentclass[12pt,a4paper,english]{article}
\usepackage{helvet}         % set font to helvetica
\renewcommand{\familydefault}{\sfdefault}
\usepackage{sectsty}        % allow redefinition of section command formatting
    \allsectionsfont{\normalsize}
\usepackage[centering,noheadfoot,margin=1in]{geometry}
\usepackage{lipsum}
\setlength{\parindent}{0em}
\usepackage{parskip}
\usepackage{hanging}
\usepackage{hyperref}
\hypersetup{colorlinks=true, linkcolor=cyan, urlcolor=cyan}
\usepackage{amsmath}
\usepackage{esvect}
\usepackage{currfile}
\setcounter{secnumdepth}{0}

\newcommand{\degree}[1]{${#1}^\circ$}

% https://tex.stackexchange.com/questions/30720/footnote-without-a-marker
\newcommand\blindfootnote[1]{
  \begingroup
  \renewcommand\thefootnote{}\footnote{#1}
  \addtocounter{footnote}{-1}
  \endgroup
}
\AtEndDocument{\blindfootnote{\textrm{This document proudly made using \LaTeX{}}}}

\begin{document}
\begin{flushright}
  Dean Gadberry

  NCTC PHYS 2425

  \today
\end{flushright}
\begin{center}
  {\large Chapter 4 Homework}
\end{center}
\begin{flushleft}

  \section*{Questions}
  \hrule
  \subsection{1. Why does a child in a wagon seem to fall backward when you give the wagon a sharp pull forward?}
  The child does not move with the wagon, because he is in another reference frame.
  \subsection{3. If the acceleration of an object is zero, are no forces acting on it? Explain.}
  The fact that an object is not accelerating is in no way correlated to whether forces are acting on it. Chairs, for example have a normal force acting upon them, and are thus, not accelerating in reference to the Earth.
  \subsection{7. If you walk on a log floating on a lake, why does the log move in the opposite direction?}
  Newton's Third Law is plain to see in this example. Each time your foot is pushing against the log, the log is pushing back and this net force along with your shifting balance moves the log in the direction which your foot pushes it.
  \subsection{16. Whiplash sometimes results from an automobile accident when the victim's car is struck violently from the rear. Explain why the head of the victim seems to be thrown backward in this situation. Is it really?}
  The victim is moving with the car, but when they are struck, their head may stay in the same place, in reference to the Earth, while the rest of their body moves forward. This phenomenon may feel like the head is moving back, but that is illusory, given the car as a reference frame.
  \section*{Misconception Questions}
  \hrule
  \subsection{2. George, in the foreground, is able to move the large truck because}
  (e) he exerts a greater force on the truck than the truck exerts back on him.
  \subsection{4. What causes the boat to move forward?}
  (a) The force the man exerts on the paddle.
  \subsection{7. A golf ball is hit with a golf club. While the ball flies through the air, which forces act on the ball? Neglect air resistance.}
  (e) Both the force of the golf club acting on the ball and the force of gravity acting on the ball.
  \subsection{13. The normal force on an extreme skier descending a very steep slope can be zero if}
  (b) he leaves the slope (no longer touches the snow)
  \section*{Problems}
  \hrule
  \subsection{2. What is the weight of a 74-kg astronaut...?}
  \subsubsection{(a) on Earth (g=9.8m/s^2)}
  \subsubsection{(b) on the Moon (g=1.7m/s^2)}
  \subsubsection{(c) on Mars (g=3.7m/s^2)}
  \subsubsection{(d) in outer space traveling with constant velocity}
  \subsection{3. How much tension must a rope withstand if it is used to accelerate a 1210-kg car horizontally along a frictionless surface at 1.35m/s^2?}
  \subsection{5. What average force is required to stop a 950-kg car in 8.04 if the car is traveling at 95km/h?}
  \subsection{8. A person has a reasonable chance of surviving an automobile crash if the deceleration is no more than 30 g's. Calculate the force on a 65-kg person accelerating at this rate. What distance is traveled if brought to rest at this rate from 85km/h?}
  \subsection{17. A woman stands on a bathroom scale in a motionless elevator. When the elevator begins to move, the scale briefly reads only 0.75 of her regular weight. Calculate the acceleration of the elevator, and find the direction of acceleration.}
  \subsection{42. A 3.0-kg object has the following two forces acting on it:}
  \[
    \vv{F}_1=(16\hat{i}+12\hat{j})N
  \]
  \[
    \vv{F}_2=(-10\hat{i}+22\hat{j})N
  \]
  \subsubsection{If the object is initially at rest, determince its velocity $\vv{v}$ at $t=4.0s$.}
  \subsection{43. A 27-kg chandelier hands from a ceiling on a vertical 3.4-m-long wire.}
  \subsubsection{(a) What horizontal force would be necessary to displace its position 0.15 m to one side?}
  \subsubsection{(b) What will be the tension in the wire?}
  \subsection{45. The block shown has mass $m=7.0$ kg and lies on a fixed smooth frictionless plane tilted at an angle $\Theta=22.0^\circ$ to the horizontal.}
  \subsubsection{(a) Determine the acceleration of the block as it slides down the plane.}
  \subsubsection{(b) If the block starts from rest 12.0 m up the plane from its base, what will be the block's speed when it reaches the bottom of the incline?}
\end{flushleft}
\end{document}
