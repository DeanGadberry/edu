% Copyright (C) Dean Gadberry - All Rights Reserved
% Unauthorized copying of this file, via any medium is strictly prohibited
% Proprietary and confidential
% Written by Dean Gadberry <dean@deangadberry.com>, 2023

\documentclass[12pt,a4paper,english]{article}
\usepackage{helvet}         % set font to helvetica
\renewcommand{\familydefault}{\sfdefault}
\usepackage{sectsty}        % allow redefinition of section command formatting
    \allsectionsfont{\normalsize}
\usepackage[centering,noheadfoot,margin=1in]{geometry}
\usepackage{lipsum}
\setlength{\parindent}{0em}
\usepackage{parskip}
\usepackage{hanging}
\usepackage{hyperref}
\hypersetup{colorlinks=true, linkcolor=cyan, urlcolor=cyan}
\usepackage{amsmath}
\usepackage{esvect}
\usepackage{currfile}
\setcounter{secnumdepth}{0}

\newcommand{\degree}[1]{${#1}^\circ$}

% https://tex.stackexchange.com/questions/30720/footnote-without-a-marker
\newcommand\blindfootnote[1]{
  \begingroup
  \renewcommand\thefootnote{}\footnote{#1}
  \addtocounter{footnote}{-1}
  \endgroup
}
\AtEndDocument{\blindfootnote{\textrm{This document proudly made using \LaTeX{}}}}

\begin{document}
\begin{flushright}
  Dean Gadberry

  NCTC PHYS 2425

  \today
\end{flushright}
\begin{center}
  {\large Chapter 8 Homework}
\end{center}
\begin{flushleft}

  \section*{Questions}
  \hrule
  \subsection{6. Experienced hikers prefer to step over a fallen log in their path rather than stepping on top and jumping down on the other side. Explain.}
  Stepping onto a log would require work (W). Stepping on a level plane but farther does not.
  \section*{Misconception Questions}
  \hrule
  \subsection{2. A bowling ball is dropped from a height h onto the center of a trampoline, which launches the ball back up into the air. How high will the ball rise?}
  (c) No more than h - probably a little less.
  \subsection{7. A skier starts from rest at the top of each of the hills shown. On which hill will the skier have the highest speed at the bottom if we ignore friction?}
  (e) c and d equally.
  \subsection{8. Answer Misconceptual Question 7 assuming a small amount of friction.}
  (c)
  \subsection{9. A child's swing is pulled back by an angle $\Theta$ and released from rest. If the child on the swing remains motionless (relative to the swing seat), what is the maximum angle the swing can reach on the other side? Neglect air resistance and friction.}
  (c) The exact same angle, $\Theta$.
  \section*{Problems}
  \hrule
  \subsection{1. By how much does the gravitational potential energy of a 58-kg pole vaulter change if her center of mass rises 4.0m during the jump?}
  \begin{align*}
  \end{align*}
  \subsection{12. Jane, looking for Tarzan, is running at top speed (5.0m/s) and grabs a vine hanging vertically from a tall tree in the jungle. How high can she swing upward? Does the length of the vine affect your answer?}
  \begin{align*}
  \end{align*}
  \subsection{14. In the high jump, the kinetic energy of an athlete is transformed into gravitational potential energy without the aid of a pole. With what minimum speed must the athlete leave teh ground in order to life his center of mass 2.10m and cross the bar with a speed of 0.50m/s?}
  \begin{align*}
  \end{align*}
  \subsection{18. A 62/kg trampoline artist jumps upward from the top of a platform with a vertical speed of 4.5m/s.}
  \subsubsection{(a) How fast is he going as he lands on the trampoline, 2.0m below?}
  \begin{align*}
  \end{align*}
  \subsubsection{(b) If the trampoline behaves like a spring of spring constant $5.8\times10^4N/m$, how far does he depress it?}
  \begin{align*}
  \end{align*}
  \subsection{28. A 16.0-kg child descends a slide 2.20m high and, starting from rest, reaches the bottom with a speed of 1.15m/s. How much thermal energy due to friction was generated in this process?}
  \begin{align*}
  \end{align*}
  \subsection{30. A 145-g baseball is dropped from a tree 14.0m above the ground}
  \subsubsection{(a) If the coefficient of friction is 0.094, what is the ski's speed at the base of the incline?}
  \begin{align*}
  \end{align*}
  \subsubsection{(b) If the snow is level at the foor of the incline and has the same coefficient of friction, how far will the ski travel along the level? Use energy methods.}
  \begin{align*}
  \end{align*}
  \subsection{55. What is the average power output of an elevator that lifts 845 kg a vertical height of 32.0m in 11.0s?}
  \begin{align*}
  \end{align*}
  \subsection{56. How long will it take a 1750-W motor to lift a 335-kg piano to a sixth-story window 18.0m above?}
  \begin{align*}
  \end{align*}
  \subsection{57. An 85-kg football player traveling 5.0m/s is stopped in 1.0s by a tackler.}
  \subsubsection{(a) What is the original kinetic energy of the player?}
  \begin{align*}
  \end{align*}
  \subsubsection{(b) What average power is required to stop him?}
  \begin{align*}
  \end{align*}
  \subsection{58. If a car generates 18hp when traveling at a steady 95km/h, what must be the average force exerted on the car due to friction and air resistance?}
  \begin{align*}
  \end{align*}
  \subsection{64. During a workout, football players ran up the stadium stairs in 75s. The distance along the stairs is 83m and they are inclined at a \degree{33} angle. If a player has a mass of 88kg, estimate his average power output on the way up. Ignore friction and air resistance.}
  \begin{align*}
  \end{align*}
  \subsection{65. A pump lifts 27.0kg of water per minute through a height of 3.50m. What minimum output rating (watts) must the pump motor have?}
  \begin{align*}
  \end{align*}
\end{flushleft}
\end{document}
