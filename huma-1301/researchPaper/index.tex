% Copyright (C) Dean Gadberry - All Rights Reserved
% % Unauthorized copying of this file, via any medium is strictly prohibd
% Proprietary and confidential
% Written by Dean Gadberry <dean@deangadberry.com>, 2023

\documentclass[12pt,a4paper,english]{article}
\usepackage{helvet}         % set font to helvetica
\renewcommand{\familydefault}{\sfdefault}
\usepackage{sectsty}        % allow redefinition of section command formatting
    \allsectionsfont{\normalsize}
\usepackage[centering,noheadfoot,margin=1in]{geometry}
\usepackage{lipsum}
\setlength{\parindent}{0.5in}
\usepackage{hanging}

\usepackage{setspace}
\doublespacing
\usepackage{indentfirst}

% \usepackage{showframe} % use to see borders of margins and such
                         % this can also be used in \usepackage[showframe]{geometry}

% https://tex.stackexchange.com/questions/30720/footnote-without-a-marker
\newcommand\blindfootnote[1]{
  \begingroup
  \renewcommand\thefootnote{}\footnote{#1}
  \addtocounter{footnote}{-1}
  \endgroup
}
\AtEndDocument{\blindfootnote{\textrm{This document proudly made using \LaTeX{}}}}

\begin{document}
\begin{titlepage}
  \begin{center}
    \vspace*{\fill}
    \textbf{The Enuma Elish: Perspectives on the World, Culture, and War}
    \par
    Exploring the Babylonian Creation Myth and Its Meaning
    \par
    Dean Gadberry
    \vspace*{\fill}
    \hfill

       HUMA 1301 0312 - The Ancients 
       \par
       Professor Ariana Warren
       \par
       North Central Texas College
       \par
       \today
  \end{center}
\end{titlepage}
\pagenumbering{gobble} %removes page numbering
\newpage
\pagenumbering{arabic} %page numbering starts on the next page
\sloppy

\section*{\centering{The Enuma Elish: Perspectives on the World, Culture, and War}}
It is fascinating that ancient civilizations viewed the world differently than modern people. But, just how much difference is there between these two groups?
The Enuma Elish sheds light on the Babylonians' perspective on the nature of the world, people, culture, art, and war.
Seven clay tablets, found in the ruins of Ninevah, hold it's secrets.
The path to understanding the origins of human civilization and how ancient peoples viewed the world begins with an exploration of the beliefs and values of ancient societies. 

The Enuma Elish was written on clay tablets between 2225 and 1926 B.C.E., during the First Babylonian Dynasty (Langon 10; Fiero 18).
It contains an Epic, "recited during the festival of the New Year" which "celebrates the birth of the gods and the order of creation" (Fiero 18).

The Enuma Elish explains that the universe was created when Marduk defeated the god of chaos, Tiamat, literally "Sea", and used her body as the substance of creation (Campbell 246; Wisnom 211).
Marduk proceeds to create humans as a means to serve the gods. 
(a) Summary of the creation myth: The Enuma Elish tells the story of the creation of the world and the emergence of the gods. 
According to the myth, the universe was created from the body of Tiamat, the primordial goddess of chaos, who was defeated by the god Marduk. 
Marduk then used Tiamat's body to create the earth and the heavens, and created humans to serve the gods.

(b) Overview of the gods: The Enuma Elish features a complex pantheon of gods and goddesses who are portrayed as powerful and capricious beings. 
The gods are divided into two main factions, led by Marduk and Tiamat, respectively, and their struggles for power and dominance are a central theme of the myth.

(c) Key themes: The Enuma Elish explores a number of themes that were important to ancient Babylonian society, including the role of the gods in human affairs, the origins of the universe, and the relationship between order and chaos. 
The myth also contains elements of political propaganda, as it presents Marduk as the supreme deity and legitimizes his authority over Babylonian society. 
Overall, the Enuma Elish offers a window into the religious beliefs and values of ancient Babylonian culture.

III. The ancient perspective on The Nature of the World

(a) The creation of the universe: The Enuma Elish provides an explanation for the origins of the universe, as well as the forces that govern it. 
According to the myth, the universe was created through a violent struggle between the forces of order and chaos, represented by the gods Marduk and Tiamat, respectively. 
The myth also describes the creation of the earth, the heavens, and the various creatures that inhabit them.

(b) What is time: The Enuma Elish also offers insights into the Babylonian concept of time. 
The myth describes a cyclical universe, where events repeat themselves in an endless cycle of creation and destruction. 
Time is seen as an important aspect of the natural order, and is governed by the movements of the celestial bodies.

(c) The role of humans and gods: The Enuma Elish portrays humans as created beings who exist to serve the gods. 
The gods are portrayed as powerful and capricious, with the ability to grant blessings or inflict punishment on humans. 
The relationship between humans and gods is complex, and is shaped by a number of factors, including worship, sacrifice, and obedience.

(d) Natural disasters: The Enuma Elish also reflects the Babylonian belief in the power of natural forces. 
The myth portrays natural disasters such as floods, earthquakes, and storms as the result of divine intervention, and suggests that they are often punishment for human disobedience or arrogance.

Overall, the Enuma Elish offers a rich and complex perspective on the nature of the world, and provides valuable insights into the Babylonian understanding of cosmology and the natural order.

IV. The ancient perspective on People

(a) Humanity and the human condition: The Enuma Elish portrays humans as fallible beings, prone to making mistakes and subject to the whims of the gods. 
The myth suggests that humans are created to serve the gods, but are also given free will to make their own choices. 
Human life is seen as a brief and fleeting existence, overshadowed by the power and eternity of the gods.

(b) Health and disease: The Enuma Elish also reflects the Babylonian belief in the importance of health and well-being. 
The myth describes a number of illnesses and afflictions, which are often seen as the result of divine punishment or intervention. 
Healing is seen as a gift from the gods, and is often achieved through prayer, sacrifice, and the use of medicinal herbs.

(c) Justice, morality, and social order: The Enuma Elish provides insights into Babylonian concepts of justice and morality. 
The myth suggests that the gods are responsible for upholding justice and maintaining order in the world, but also acknowledges the role of humans in creating and enforcing laws. 
Moral values such as honesty, loyalty, and humility are seen as important, but are often overshadowed by the power and influence of the gods.

(d) The afterlife and soul: The Enuma Elish also touches on Babylonian beliefs about the afterlife and the fate of the soul. 
The myth suggests that the soul lives on after death, and is judged by the gods based on its deeds in life. 
The afterlife is portrayed as a shadowy and uncertain realm, where the soul is subject to the whims of the gods.

Overall, the Enuma Elish offers a nuanced and complex perspective on the human condition, and provides valuable insights into Babylonian concepts of health, morality, and the afterlife.

V. The ancient perspective on Art and Culture

(a) Purpose and function of art: The Enuma Elish sheds light on the Babylonian view of art and its function in society. 
The myth suggests that art was primarily used for religious and ritualistic purposes, and was often commissioned by kings and priests to honor the gods. 
Art was seen as a way to communicate with the divine, and was believed to have the power to invoke the gods and protect the community from harm.

(b) What is beautiful, creation, innovation?: The Enuma Elish also reveals the Babylonian aesthetic sense and their appreciation for beauty and craftsmanship. 
The myth describes intricate and detailed depictions of the gods, as well as ornate temples and palaces. 
The Babylonians placed great value on creativity and innovation, and were known for their advances in fields such as astronomy and mathematics.

(c) Religious and mythological art: The Enuma Elish contains many examples of religious and mythological art, which were intended to tell the stories of the gods and their interactions with humanity. 
These artworks often featured intricate symbols and motifs, and were believed to have the power to protect and bless those who viewed them.

Overall, the Enuma Elish provides valuable insights into Babylonian art and culture, and highlights the importance of creativity, beauty, and religious expression in ancient society.

VI. The ancient perspective on War

(a) Causes of war: The Enuma Elish provides a glimpse into the Babylonian understanding of the causes of war. 
According to the myth, conflicts among the gods were the primary cause of wars on earth. 
The gods were often depicted as fickle and capricious, and their disputes frequently spilled over into human affairs. 
Additionally, economic and political factors, such as resource scarcity and territorial disputes, also played a role in the outbreak of wars.

(b) Ethics of war: The Babylonians had a complex set of ethical beliefs surrounding warfare. 
While war was seen as a necessary and often justifiable means of protecting one's community or acquiring resources, certain actions were considered unethical or taboo. 
For example, attacking unarmed civilians or desecrating religious sites was viewed as a violation of the gods' laws and could bring divine retribution.

(c) Religious and mythological roles in justifying and condemning combat: The Enuma Elish portrays the gods as both supporters and opponents of warfare. 
Some gods, such as Marduk, are depicted as warrior deities who actively participate in battles and provide aid to their favored humans. 
Other gods, however, are more peaceable and seek to prevent conflict or punish those who engage in unjust wars. 
This duality highlights the complex role that religion and mythology played in justifying and condemning combat in Babylonian society.

In conclusion, the Enuma Elish offers valuable insights into the Babylonian perspective on war and sheds light on the complex ethical and religious beliefs surrounding armed conflict in ancient times.

VII. Conclusion

The Enuma Elish is a fascinating creation myth that provides a window into the ancient Babylonian perspective on the world, people, culture, art, and war. 
Through an analysis of the myth, we have seen that the Babylonians had a complex understanding of the nature of the world, which they believed was created through a violent struggle between the gods. 
They also had a nuanced view of humanity and the human condition, which encompassed concepts such as justice, morality, and the afterlife.

Furthermore, the Babylonians had a rich artistic tradition that was closely intertwined with their religious beliefs and mythological stories. 
Finally, the Enuma Elish reveals the Babylonians' complex attitudes towards war, including the causes of conflict, the ethics of war, and the role of religion and mythology in justifying or condemning combat.

Overall, the Enuma Elish provides a fascinating glimpse into the ancient Babylonian worldview, and its insights remain relevant today for those interested in the history of ideas and the development of human thought. 
By studying the Enuma Elish, we can gain a deeper appreciation for the diversity and complexity of human culture and gain insights into the ways in which different societies have grappled with fundamental questions about the world and the human experience.

\newpage
\section*{\centering{Works Cited}}
\begin{hangparas}{2em}{1}
  Campbell, J. (2008). \emph{The Hero with a Thousand Faces}. New World Library.
  \par
  Fiero, G. (2010). \emph{The Humanistic Tradition, Book 1: The First Civilizations and the Classical Legacy}. McGraw-Hill Education.
  \par
  Langdon, S. (1923). \emph{The Babylonian Epic of Creation Restored From the Recently Recovered Tablets of Assur}. Clarendon Press. https://archive.org/details/babylonianepicof00languoft/page/38/mode/2up
  \par
  Peterson, Jordan., and Whanhee (Uploader). (2017, June 12). \emph{Mesopotamian Gods} [Video]. YouTube. https://www.youtube.com/watch?v=JodNMjXphKA
  \par
Tugendhaft, Aaron. “Politics and Time in the Baal Cycle.” \emph{Journal of Ancient Near Eastern Religions}, vol. 12, no. 2, Sept. 2012, pp. 145–57. EBSCOhost, https://doi-org.northcenttexascollegelibrary.idm.oclc.org/10.1163/15692124-12341235.
  \par
  Weeks, Noel K. “Myth and Ritual: An Empirical Approach.” \emph{Journal of Ancient Near Eastern Religions}, vol. 15, no. 1, Jan. 2015, pp. 92–111. EBSCOhost, https://doi-org.northcenttexascollegelibrary.idm.oclc.org/10.1163/15692124-12341270.
  \par
Wisnom, Selena. “Marduk the Fisherman.” \emph{Journal of the American Oriental Society}, vol. 141, no. 1, Jan. 2021, pp. 211–14. EBSCOhost, https://doi-org.northcenttexascollegelibrary.idm.oclc.org/10.7817/jameroriesoci.141.1.0211.
\end{hangparas}
\par
\fussy
\end{document}

\iffalse
Langdon
    10:
        written period of First Babylonian Dynasty
        monsters of Chaos which Marduk subdued in his combat with Tiamat
    12: Bk I 1-20 In the beginning, only Apsu, the fresh water ocean and tiamat, the salt ocean existed. They were mingled in one. From the union of the male Apsu and the dragon of Chaos, Tiamat, the pair Lahmu and Lahamu were engedered, and after many ages, Ansar and Kisar came into being. These two deities are the first of the gods of order, and the engendered Any the heaven god and Ea the water god.
        Bk I 21-28 lesser gods rebelled
    13: Bk II 120-9 Marduk demands promotion to the rank of a great god as a reward for his bravery and under the condition that he is able to defeat Tiamat
    14: Bk III 125-38 Tiamat had conspired to destroy the gods.
        Bk IV 1-18 the founded a chamber for Marduk in the Hall of Fates and he is thus added to the sacred assembly of the highest gods. He receives the power to declare fates and work miracles
    15: Bk IV 71-134 Defeat of Tiamat
        Bk IV 135-46 Marduk fixes the abode of the three gods of the trinity
        Bk V Astronomical movements: planets, moon, signs of the zodiac
        Bk VI 29-35 Marduk divides two groups of gods: those 600 of the upper world and heavens, and the 50 Annunaki of the lower world.
        Bk VI 36-41 In gratitude, gods decide to build a great shrine on earth for Marduk for New years assemblies (this is Babylon)
    16: Bk VI 65-8 Marduk's bow is fixed in the sky as Canis Major
        The Epic ends with a scene of the Babylonian celebration of the New year's festival, which is held during the spring equinox

Tugendhaft
    148: "both poems center on a god who is victorious in battle; in both the victorious deity i sproclaimed a king; both tell of a royal abode being built for that god."
    152: "Tiamat belongs to the first generation of primordial gods and Marduk is among the youngest of the active deities."

Weeks
    95: The evidence that the Babylonian rituals are connected to their myths come from sources other than the Enuma Elish.

Peterson:
(00:15)How they viewed the world: Disk of salt water, on top of it is a dome. It is phenomenological representation. The dome of the land is on a disk of fresh water on a disk of salt water. Tiamat is god of salt water.(female) Apsu is god of fresh water.(male) They are locked together and their union created child gods. 

(1:40)The child gods represent primary modes of being.(simple perspectives of the world) 
    Why are they deities? 
        They live forever, 
        They control you, 
        They are personalities.

    Humans are the playthings of the gods (2:37)

(Fiero 18) Elder gods make their home on the corpse of Apsu. 
People inhabit the corpse of culture (the dead past). (4:00)
Tiamat awakes and creates an army of [chimeric] monsters led by Kingu (like Satan).(6:20)
Multiple gods are sent to destroy Tiamat. (7:00) None succeed. 
Marduk has the capacity to speak, and as the hero, this identifies speech as the advancing force of heirarchy dominance. (7:45)
Marduk uses a net, which represents the ability to subdivide a chaotic concept for ease of management. (9:30)
He cuts Tiamat into pieces and makes the world.
Humans are created out of the blood of Kingu (10:00) This is why people are capable of deception.
11:20 Emperor is a manifestation of Marduk. He is king because they pay attention, speak properly, keep chaos at bay, and make ingenious things happen.
11:45 The mesopotamians are answering the question "What should be sovereign, and why?"
The New Year Festival is when the mesopotamians remind each other that leadership requires these qualities. (12:45)

Fiero
18 "The Babylonian Creation is a Sumerian poem recorded in the second millenium BCE. Recited during the festival of the New Year, it celebrates the birth of the gods and the order of creation"

\fi
