% Copyright (C) Dean Gadberry - All Rights Reserved
% % Unauthorized copying of this file, via any medium is strictly prohibd
% Proprietary and confidential
% Written by Dean Gadberry <dean@deangadberry.com>, 2023

\documentclass[12pt,a4paper,english]{article}
\usepackage{helvet}         % set font to helvetica
\renewcommand{\familydefault}{\sfdefault}
\usepackage{sectsty}        % allow redefinition of section command formatting
    \allsectionsfont{\normalsize}
\usepackage[centering,footskip=.45in,margin=1in]{geometry}
\usepackage{lipsum}
\setlength{\parindent}{0.5in}
\usepackage{hanging}

\usepackage{setspace}
\doublespacing
\usepackage{indentfirst}

% \usepackage{showframe} % use to see borders of margins and such
                         % this can also be used in \usepackage[showframe]{geometry}

% https://tex.stackexchange.com/questions/30720/footnote-without-a-marker
\newcommand\blindfootnote[1]{
  \begingroup
  \renewcommand\thefootnote{}\footnote{#1}
  \addtocounter{footnote}{-1}
  \endgroup
}
\AtEndDocument{\blindfootnote{\textrm{This document proudly made using \LaTeX{}}}}

\begin{document}
\begin{titlepage}
  \begin{center}
    \vspace*{\fill}
    \textbf{The Enuma Elish: Perspectives on the World, Culture, and Leadership}
    \par
    Exploring the Babylonian Creation Myth and Its Meaning
    \par
    Dean Gadberry
    \vspace*{\fill}
    \hfill

       HUMA 1301 0312 - The Ancients 
       \par
       Professor Ariana Warren
       \par
       North Central Texas College
       \par
       \today
  \end{center}
\end{titlepage}
\pagenumbering{gobble} %removes page numbering
\newpage
\pagenumbering{arabic} %page numbering starts on the next page
\sloppy

\section*{\centering{The Enuma Elish: Perspectives on the World, Culture, and Leadership}}
The path to understanding the origins of human civilization and how ancient peoples viewed the world begins with an exploration of the beliefs and values of ancient societies. 
It is fascinating that ancient civilizations viewed the world differently than modern people. But, just how much difference is there between these two groups?
The Enuma Elish sheds light on the Babylonians' perspective on the nature of the world, people, culture, war, and leadership.
Seven clay tablets, found in the ruins of Ninevah, hold it's secrets.

The Enuma Elish was written between 2225 and 1926 B.C.E., during the First Babylonian Dynasty (Langdon 10; Fiero 18). 
It contains an Epic, "recited during the [Babylonian] festival of the New Year" which "celebrates the birth of the gods and the order of creation" (Fiero 17).

The Enuma Elish explains that the universe was created when the god Marduk defeated the god of chaos, Tiamat, literally "Sea", and used her body as the substance of creation (Campbell 246; Wisnom 211). 

It begins by describing the development of the gods. 
The primeval gods, Apsu and Tiamat, mingle to give rise to another generation of gods, known as the elder gods. 
The elder gods foolishly kill Apsu. 
They make their home on the corpse of the primeval father and desecrate his image. 

The angered Tiamat creates an army of monsters to avenge her husband's death. 
She creates nine destructive chimeras to wage war against the insurrection. 
She marries a new created god, Kingu. 
He is the leader of Tiamat's monster army. 
The king of the demons.

The elder gods send many gods to fight against Tiamat to no avail. 
When Marduk is sent to fight Tiamat, he makes a deal with the gods. 
He is granted great abilities and the power to determine fates, under the condition that he defeats Tiamat.
Marduk casts a net upon Tiamat and cuts her body apart.
He creates the universe from the body of Tiamat and creates humans from the blood of Kingu. 

The story ends with the celebration of New Years, a time of renewal.

The Enuma Elish also describes the movements of celestial bodies at length. The Babylonians show a great understanding of what is happening in the sky. They believe, even, that the bow of Marduk, used to kill Tiamat, was placed in the heavens for remembrance. They attribute Canis Major with this honor. (Enuma Elish Bk VI 65-8)

Jordan Peterson recognizes the "phenomenological representation" of the world which the Babylonians adopted. 
In the Enuma Elish, the world is described as a "dome on a disc" (Peterson 00:45).
This simplistic understanding shows the lack of scientific discoveries which the Babylonians had yet to find out. 
They believed that the dome of salt water which extended very far had a lang mass, under which fresh water was found.
The primeval gods were these water forces which made life. 

The elder gods represent ways in which people operate, ways in which people live. 
They are identifiable and simplistic. 
Single-faceted.
They are in rebellion against the natural world.
They live on the corpse of Apsu, fighting off the chaos of natural order.
Humans, likewise, live on the fresh water browned by dirt and grime.
This anthropomorphization shows the Babylonian's understanding of the primary modes of operating as a human being.

Tiamat's anger towards the elder gods shows the natural world's attempt to squash humans. 
The planet is fighting to ruin and destroy the human endeavor.
The beasts of land are congruent with the beasts of Tiamat. 

Tugendhaft remarks that this epic centers "on a god who is victorious in battle", that he "is proclaimed a king", and it tells of a "royal abode being built for that god" (Tugendhaft 148).
Marduk enters the scene as a unique individual.
He is capable of confronting chaos.
He ultimately defeats it.
He divides Tiamat up into manageable pieces and creates a world upon it.
This is the ability of a leader to create society from such a difficult situation. 

So, this Marduk character must be superposable with the leader of the Babylonians.
His skills are speaking and seeing (Peterson 8:10). 
He, like Marduk is in control and makes a path for a society to flourish.
The leader of a society is tasked with the position of dealing with the chaos and consistently confronting and responding to it.
He must be able to destroy the beasts and make use of them, even the most ferocious beasts, such as Kingu represents in the epic. 

The people, of course, must keep their leader accountable to their desires. 
The Enuma Elish story ends with the scene of the renewal of the year. 
The Babylonians are intent to remind their king and themselves of what is expected of all parties. 
Thus, the New Year's Festival, in Babylon, was a time in which this story was reenacted by the priests. 
They would gather on the spring solstice and renew their intention to act in the best interest of society (Peterson 12:45).

Weeks remarks that the evidence that the Babylonian rituals are connected to their myths come from sources other than the Enuma Elish.
Of course, these other sources are the same which document further history of the Babylonians.
It would be curious to see what these sources have to say regarding the Babylonians' understanding of the sky.

In the Enuma Elish, the sky is mapped poetically. 
In Book 5, the movements of the planets are documented vehemently. 
The movements of the sky must have been very important to the Babylonians.
It would inform the priests of events happening between the gods.
The priests could use this information to teach the people about incoming disasters, and help counsel the Babylonian leader regarding his actions.
He is, as has been noted, supposed to model Marduk in his every step.
What better way than with a map of the stars and a cabinet of expert astronomers?

The Enuma Elish opens a door into the mind of a Babylonian living thousands of years ago. 
It makes light of their understanding of the world and society. 
The struggle between chaos and order is evident above all else. 
Their recognition of good leadership shows that the culture, then, was intensly advanced. 
There is powerful metaphor in Marduk's victory and creation of a habitable world. 
It models, well, the role of a good leader.

Not only this, but the Babylonian's had great awareness of their connection to the natural world. 
They were capable of using signs in the natural world to influence the outcome of humanity.

Overall, this document reveals the Babylonians' perspective regarding the world, culture, and leadership.
It has much to offer our society today.
Any leader, today, could benefit from a healthy analysis of the leadership skills presented thousands of years ago, in the ancient near east.

\newpage
\section*{\centering{Works Cited}}
\begin{hangparas}{2em}{1}
  Campbell, J. (2008). \emph{The Hero with a Thousand Faces}. New World Library.
  \par
  Fiero, G. (2010). \emph{The Humanistic Tradition, Book 1: The First Civilizations and the Classical Legacy}. McGraw-Hill Education.
  \par
  Langdon, S. (1923). \emph{The Babylonian Epic of Creation Restored From the Recently Recovered Tablets of Assur}. Clarendon Press. https://archive.org/details/babylonianepicof00languoft/page/38/mode/2up
  \par
  Peterson, Jordan., and Whanhee (Uploader). (2017, June 12). \emph{Mesopotamian Gods} [Video]. YouTube. https://www.youtube.com/watch?v=JodNMjXphKA
  \par
Tugendhaft, Aaron. “Politics and Time in the Baal Cycle.” \emph{Journal of Ancient Near Eastern Religions}, vol. 12, no. 2, Sept. 2012, pp. 145–57. EBSCOhost, https://doi-org.northcenttexascollegelibrary.idm.oclc.org/10.1163/15692124-12341235.
  \par
  Weeks, Noel K. “Myth and Ritual: An Empirical Approach.” \emph{Journal of Ancient Near Eastern Religions}, vol. 15, no. 1, Jan. 2015, pp. 92–111. EBSCOhost, https://doi-org.northcenttexascollegelibrary.idm.oclc.org/10.1163/15692124-12341270.
  \par
Wisnom, Selena. “Marduk the Fisherman.” \emph{Journal of the American Oriental Society}, vol. 141, no. 1, Jan. 2021, pp. 211–14. EBSCOhost, https://doi-org.northcenttexascollegelibrary.idm.oclc.org/10.7817/jameroriesoci.141.1.0211.
\end{hangparas}
\par
\fussy
\end{document}

\iffalse
Langdon
    10:
        written period of First Babylonian Dynasty
        monsters of Chaos which Marduk subdued in his combat with Tiamat
    12: Bk I 1-20 In the beginning, only Apsu, the fresh water ocean and tiamat, the salt ocean existed. They were mingled in one. From the union of the male Apsu and the dragon of Chaos, Tiamat, the pair Lahmu and Lahamu were engedered, and after many ages, Ansar and Kisar came into being. These two deities are the first of the gods of order, and the engendered Any the heaven god and Ea the water god.
        Bk I 21-28 lesser gods rebelled
    13: Bk II 120-9 Marduk demands promotion to the rank of a great god as a reward for his bravery and under the condition that he is able to defeat Tiamat
    14: Bk III 125-38 Tiamat had conspired to destroy the gods.
        Bk IV 1-18 the founded a chamber for Marduk in the Hall of Fates and he is thus added to the sacred assembly of the highest gods. He receives the power to declare fates and work miracles
    15: Bk IV 71-134 Defeat of Tiamat
        Bk IV 135-46 Marduk fixes the abode of the three gods of the trinity
        Bk V Astronomical movements: planets, moon, signs of the zodiac
        Bk VI 29-35 Marduk divides two groups of gods: those 600 of the upper world and heavens, and the 50 Annunaki of the lower world.
        Bk VI 36-41 In gratitude, gods decide to build a great shrine on earth for Marduk for New years assemblies (this is Babylon)
    16: Bk VI 65-8 Marduk's bow is fixed in the sky as Canis Major
        The Epic ends with a scene of the Babylonian celebration of the New year's festival, which is held during the spring equinox

Tugendhaft
    148: "both poems center on a god who is victorious in battle; in both the victorious deity i sproclaimed a king; both tell of a royal abode being built for that god."
    152: "Tiamat belongs to the first generation of primordial gods and Marduk is among the youngest of the active deities."

Weeks
    95: The evidence that the Babylonian rituals are connected to their myths come from sources other than the Enuma Elish.

Peterson:
(00:15)How they viewed the world: Disk of salt water, on top of it is a dome. It is phenomenological representation. The dome of the land is on a disk of fresh water on a disk of salt water. Tiamat is god of salt water.(female) Apsu is god of fresh water.(male) They are locked together and their union created child gods. 

(1:40)The child gods represent primary modes of being.(simple perspectives of the world) 
    Why are they deities? 
        They live forever, 
        They control you, 
        They are personalities.

    Humans are the playthings of the gods (2:37)

(Fiero 18) Elder gods make their home on the corpse of Apsu. 
People inhabit the corpse of culture (the dead past). (4:00)
Tiamat awakes and creates an army of [chimeric] monsters led by Kingu (like Satan).(6:20)
Multiple gods are sent to destroy Tiamat. (7:00) None succeed. 
Marduk has the capacity to speak, and as the hero, this identifies speech as the advancing force of heirarchy dominance. (7:45)
Marduk uses a net, which represents the ability to subdivide a chaotic concept for ease of management. (9:30)
He cuts Tiamat into pieces and makes the world.
Humans are created out of the blood of Kingu (10:00) This is why people are capable of deception.
11:20 Emperor is a manifestation of Marduk. He is king because they pay attention, speak properly, keep chaos at bay, and make ingenious things happen.
11:45 The mesopotamians are answering the question "What should be sovereign, and why?"
The New Year Festival is when the mesopotamians remind each other that leadership requires these qualities. (12:45)

Fiero
18 "The Babylonian Creation is a Sumerian poem recorded in the second millenium BCE. Recited during the festival of the New Year, it celebrates the birth of the gods and the order of creation"

\fi
