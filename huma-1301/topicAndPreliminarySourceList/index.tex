% Copyright (C) Dean Gadberry - All Rights Reserved
% % Unauthorized copying of this file, via any medium is strictly prohibd
% Proprietary and confidential
% Written by Dean Gadberry <dean@deangadberry.com>, 2023

\documentclass[12pt,a4paper,english]{article}
\usepackage{helvet}         % set font to helvetica
\renewcommand{\familydefault}{\sfdefault}
\usepackage{sectsty}        % allow redefinition of section command formatting
    \allsectionsfont{\normalsize}
\usepackage[centering,noheadfoot,margin=1in]{geometry}
\usepackage{lipsum}
\setlength{\parindent}{0em}
\usepackage{hanging}

% https://tex.stackexchange.com/questions/30720/footnote-without-a-marker
\newcommand\blindfootnote[1]{
  \begingroup
  \renewcommand\thefootnote{}\footnote{#1}
  \addtocounter{footnote}{-1}
  \endgroup
}
\AtEndDocument{\blindfootnote{\textrm{This document proudly made using \LaTeX{}}}}

\begin{document}
\sloppy
Dean Gadberry
\par
HUMA 1301 0312
\par
2/9/2023

\section*{\centering{Topic and Preliminary Sources}}
\textbf{Topic:}
What does a study of the Enuma Elish reveal about the ancient perspective on the nature of the world, people, culture, art, and war?

\section*{Primary Sources:}
\begin{hangparas}{2em}{1}
  Langdon, S. (1923). \emph{The Babylonian Epic of Creation Restored From the Recently Recovered Tablets of Assur}. Clarendon Press. https://archive.org/details/babylonianepicof00languoft/page/38/mode/2up
  \par
\end{hangparas}
\hfill

\section*{Secondary Sources:}
\begin{hangparas}{2em}{1}
  Peterson, Jordan., and Whanhee (Uploader). (2017, June 12). \emph{Mesopotamian Gods} [Video]. YouTube. https://www.youtube.com/watch?v=JodNMjXphKA
  \par
  Helle, Sophus. “The Two-Act Structure: A Narrative Device in Akkadian Epics.” \emph{Journal of Ancient Near Eastern Religions}, vol. 20, no. 2, July 2020, pp. 190–224. EBSCOhost, https://doi-org.northcenttexascollegelibrary.idm.oclc.org/10.1163/15692124-12341315.
  \par
Wisnom, Selena. “Marduk the Fisherman.” \emph{Journal of the American Oriental Society}, vol. 141, no. 1, Jan. 2021, pp. 211–14. EBSCOhost, https://doi-org.northcenttexascollegelibrary.idm.oclc.org/10.7817/jameroriesoci.141.1.0211.
  \par
  Xiang, Zairong. “Below Either/Or: Rereading Femininity and Monstrosity Inside Enuma Elish.” \emph{Feminist Theology: The Journal of the Britain & Ireland School of Feminist Theology}, vol. 26, no. 2, Jan. 2018, pp. 115–32. EBSCOhost, https://doi-org.northcenttexascollegelibrary.idm.oclc.org/10.1177/0966735017737716.
  \par
  Weeks, Noel K. “Myth and Ritual: An Empirical Approach.” \emph{Journal of Ancient Near Eastern Religions}, vol. 15, no. 1, Jan. 2015, pp. 92–111. EBSCOhost, https://doi-org.northcenttexascollegelibrary.idm.oclc.org/10.1163/15692124-12341270.
  \par
Tugendhaft, Aaron. “Politics and Time in the Baal Cycle.” \emph{Journal of Ancient Near Eastern Religions}, vol. 12, no. 2, Sept. 2012, pp. 145–57. EBSCOhost, https://doi-org.northcenttexascollegelibrary.idm.oclc.org/10.1163/15692124-12341235.
\end{hangparas}
\hfill

\section*{Book Sources:}
\begin{hangparas}{2em}{1}
  Campbell, J. (2008). \emph{The Hero with a Thousand Faces}. New World Library.
  \par
  Fiero, G. (2010). \emph{The Humanistic Tradition, Book 1: The First Civilizations and the Classical Legacy}. McGraw-Hill Education.
\end{hangparas}
\par
\fussy
\end{document}
