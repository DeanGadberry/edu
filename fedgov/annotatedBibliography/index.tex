%% This is index.tex.

% MLA Format
\documentclass [12pt]{article}
\usepackage[letterpaper]{geometry}
\geometry{top=1.0in, bottom=1.0in, left=1.0in, right=1.0in}
\usepackage{times}
\usepackage{setspace}
\doublespacing
\usepackage{rotating}

% no section numbers
\setcounter{secnumdepth}{0}

\usepackage{xurl}
\urlstyle{same}

\newcommand{\bibent}{\noindent \hangindent 40pt}
\newcommand{\bibannote}{\begin{quotation}}
\newcommand{\bibendote}{\end{quotation}}

\title{Gun Control\\\medskip An Annotated Bibliography}
\author{Dean Gadberry (school@deangadberry.com)\\North Central Texas College}

\begin{document}

\begin{titlepage}
\centering
    North Central Texas College 
\vfill
    Gun Control \par
    An Annotated Bibliography \par
\vfill
\parbox{.5\textwidth}
    {\centering 
        Dean Gadberry \par
        GOVT 2305 0230\par
        Dr. David Smith \par
        2023-01-02
    }
\end{titlepage}
\pagebreak

\section{Policy Topic}
The following articles regard the effects and efficacy of gun control or the lack thereof. Gun Control has been in the public foresight via the news media and social media presented in America today. The media portrays gun violence as highly prevalent, and therefore, many politicians have presented official opinions and stances on gun legislation. Thankfully, this has also given rise to many academic research articles on the topic of Gun Control. This is especially intriguing, as much of the research found is at odds with each other. Experts from many fields have weighed in on the topic of Gun Control and there is absolutely no consensus. \par I have found this topic extremely intriguing over the past few years, as some of the largest names in media have taken on this beast of a topic. Many believe that guns don't kill people, but rather, people kill people--That the intent to kill will drive someone to commit a murder, regardless of the law. Some people believe that guns are a necessity to a free and democratic society, and that the lack thereof would lead to a dictatorial version of what they believe to be wonderful and pure. Some, still, take an apathetic approach, aand would rather not be bothered by the subject. Regardless where one stands, the topc is in the public eye, and it will continue to be debated until a resolution which satisfies the masses comes about. Considering the unlikelihood of satisfied masses, it may be assumed that the Gun Control Debate is here to stay. 
\pagebreak
\section{Research Process}
In the process of researching this topic, I endeavored to search the annals of academic journals and articles. With the help of JSTOR and Gale, I was able to search for articles pertaining to gun control. I worked to find articles that pertained to an opinion or effect of gun control. As I am searching for the effects and efficacy of gun control, I must gauge whether the public would be satisfied with any given gun control action. I must also find articles which detail which gun control legislation can be passed in the United States, considering the presence of the 2nd amendment. Then, given this litmus test, the opinions of those suggesting legislation can be weighed against those articles which present data for or against the efficacy of gun legislation. The articles found in this process often presented diametrically opposed styles and presentations. While some are data-driven, research-oriented articles published under the name of a major university, others are opinion-based slander-pieces published by individuals or small groups. This pantheon of opinion and fact makes for good reading and better results. 
\pagebreak
\section{Works Cited}
\bibent Brown, Erin. "Gun Control: Public Perception." Gale Essential Overviews: Scholarly, Gale, 2018. Gale Academic OneFile, \url{link.gale.com/apps/doc/BZGXMF412697823/AONE?u=txshracd2531&sid=bookmark-AONE&xid=4908dfd7.}. Accessed 31 Dec. 2022.
\bibannote
This updated survey presents the American's opinion on gun violence by analyzing and regurgitating information released by Pew Research Center and the Gallup Organization. This paper has plenty of hot takes, but none that bring the gun control issue to head. Erin Brown's credentials are not presented by this journal entry, and searches on databases produce scant results. While this paper does not provide any new data for the gun control debate, it does compile data into an easy-to-read format. This paper helps us determine whether gun control efforts would satiate the appetite of the public, at large. 
\bibendote

\bibent Philip J. Cook, and Harold A. Pollack. “Reducing Access to Guns by Violent Offenders.” RSF: The Russell Sage Foundation Journal of the Social Sciences, vol. 3, no. 5, 2017, pp. 2–36. JSTOR, \url{https://doi.org/10.7758/rsf.2017.3.5.01}. Accessed 1 Jan. 2023.
\bibannote
This paper works to convince the reader that recuding gun violence is one of the most important political policy topics today. Unique insight is provided regarding the underground arms market, and possible policy decisions to limit such a market. Philip J. Cook is the Professor of Public Policy at the Sanford School of Public Policy at Duke University in the United States. He also holds faculty appointments in Duke's departments of sociology, and economics. Harold A. Pollack is the Helen Ross Professor of Public Health Sciences at the University of Chicago. These two men are experts in the field of public policy. This argumentative piece draws heavily on hypotheticals and "thought experiments" to attempt to draw the reader closer to a political goal. The efficacy of gun control, however, is shown in unique light. 
\bibendote

\bibent Erskine, Hazel. “The Polls: Gun Control.” The Public Opinion Quarterly, vol. 36, no. 3, 1972, pp. 455–69. JSTOR, \url{http://www.jstor.org/stable/2747449}. Accessed 1 Jan. 2023.
\bibannote
This piece addresses the complacency of Public officials. While the politicians do nothing to bring progress to the cries of millions of Americans, Erskine presents the public sentiment regarding action on the gun control debate. In a record Harris Survey and Gallup Poll, only ~50\% of americans are found to own a firearm. Their opinions on gun control are presented in an easy to read way. Hazel Erskine was a pioneer in audience and opinion polling. Her work stands as the foundation for modern opinion polling. This paper helps us determine whether gun control efforts would satiate the appetite of the public, at large. 
\bibendote

\bibent Helmke, Paul. “Targeting Gun Violence.” Public Administration Review, vol. 73, no. 4, 2013, pp. 551–52. JSTOR, \url{http://www.jstor.org/stable/42003075}. Accessed 1 Jan. 2023.
\bibannote
Helmke presents, in this piece, five points to help provide framework for policy deliberations. He highlights the fact that Americans own the most guns, per capita, in the world. This should mean that "If more guns made us safer, we should be the safest country in the world." After working as president of the Brady Campaign to Prevent Gun Violence, Paul Helmke served as Mayor of Fort Wayne, Indiana. He is a former president of The United States Conference of Mayors, and is currently the director of the Civic Leaders Living-Learning Center at the Indiana University School of Public \& Environmental Affairs. Helmke fails to address the very real issue of the presence of the 2nd amendment. This oversight reduces credibility and practicality of his very opinion-based writing. This work addresses some of the concerns that law-makers must confront, if gun control will or will not be put into effect. But with more opinion than fact, Helmke misses the mark.
\bibendote

\bibent Itah, Maya. "HOW THE GUN CONTROL ACT DISARMS BLACK FIREARM OWNERS." Washington Law Review, vol. 96, no. 3, Oct. 2021, pp. 1191+. Gale Academic OneFile, \url{link.gale.com/apps/doc/A683607804/AONE?u=txshracd2531&sid=bookmark-AONE&xid=d62b28b2.}. Accessed 31 Dec. 2022.
\bibannote
This article shows how federal courts' expansive interpretation of the Gun Control Act has led to the disproportionate incarceration of Black firearm owners. The article suggests that, either, the law ought to be interpreted more narrowly or that the language at issue should be removed from the law. These reforms are suggested by a person who is difficult to find. Maya Itah's name does not seem to be associated with any universities or professional institutions. The comments which she makes are narrow and seem to be driven by her personal issue with the subject at hand. This article is an up-close insight as to what an American citizen may believe ought to be done about gun restrictions. This article pertains to the efficacy of gun control.
\bibendote

\bibent Kleck, Gary, and E. Britt Patterson. “The Impact of Gun Control and Gun Ownership Levels on Violence Rates.” Journal of Quantitative Criminology, vol. 9, no. 3, 1993, pp. 249–87. JSTOR, \url{http://www.jstor.org/stable/23365752}. Accessed 1 Jan. 2023.
\bibannote
Abstract: \par What effects do gun control restrictions and gun prevalence have on rates of violence and crime? Data were gathered for all 170 U.S. cities with a 1980 population of at least 100,000. The cities were coded for the presence of 19 major categories of firearms restriction, including both state- and city-level restrictions. Multiple indirect indicators of gun prevalence levels were measured and models of city violence rates were estimated using two-stage least-squares methods. The models covered all major categories of intentional violence and crime which frequently involve guns: homicide, suicide, fatal gun accidents, robbery, and aggravated assaults, as well as rape. Findings indicate that (1) gun prevalence levels generally have no net positive effect on total violence rates, (2) homicide, gun assault, and rape rates increase gun prevalence, (3) gun control restrictions have no net effect on gun prevalence levels, and (4) most gun control restrictions generally have no net effect on violence rates. There were, however, some possible exceptions to this last conclusion—of 108 assessments of effects of different gun laws on different types of violence, 7 indicated good support, and another 11 partial support, for the hypothesis of gun control efficacy. \par
This source considers data from 170 large cities in the United States. The data shows correlations and links between violence and gun prevalence. It is interesting to note that most general gun control measures had no effect on violent crime rates. Gary Kleck is Professor Emeritus of Criminology at Florida State University. In this article, he fails to show the rates of mid-sized, and even small, towns and cities in America. These data points are crucial, as the less populated areas of America may have higher crime rates, and more impact from gun control. This work directly addresses the effects of gun control regarding large urban environments.
\bibendote


\bibent Kleck, Gary. "Gun control after Heller and McDonald:. what cannot be done and what ought to be done." Fordham Urban Law Journal, vol. 39, no. 5, Oct. 2012, pp. 1383+. Gale Academic OneFile, \url{link.gale.com/apps/doc/A324205898/AONE?u=txshracd2531&sid=bookmark-AONE&xid=9b2abc7b.}. Accessed 31 Dec. 2022.
\bibannote
Here, we see what may and may not be done to restrict gun accessibility. There is clearly a restriction on gun control: The 2nd amendment. Not much can be done to ban guns from being accessed by American Citizens. Gary Kleck is Professor Emeritus of Criminology at Florida State University. The holistic approach of this paper brings lightto the diligence and astuteness of the researchers. This paper shows where the lines of public sentiment and constitutional practicality meet, intersect, and otherwise produce fruit. 
\bibendote

\bibent Kwon, Ik-Whan G., and Daniel W. Baack. “The Effectiveness of Legislation Controlling Gun Usage: A Holistic Measure of Gun Control Legislation.” The American Journal of Economics and Sociology, vol. 64, no. 2, 2005, pp. 533–47. JSTOR, \url{http://www.jstor.org/stable/3488101}. Accessed 1 Jan. 2023.
\bibannote
Abstract: \par Results from past research on the effectiveness of gun control legislation have been mixed. This study posits that one of the reasons for these conflicting results is the use of individual laws as the major variable. Instead, this study uses a holistic and comprehensive measure of state gun control laws, grouping states into extreme and lax gun control states. A multivariate linear regression analysis is used to investigate the relationship between a set of determinants, including the holistic gun control measure, and firearm deaths per 100,000 inhabitants of each state. The results show that comprehensive gun control legislation indeed lowers the number of gun-related deaths anywhere between one to almost six per 100,000 individuals in those states that have the most extreme gun-related legislation. Our study also reveals that socioeconomic and law enforcement factors play equally important roles in containing gun-related fatalities. These findings suggest that gun-related deaths have a variety of causes and that attempts to legislate a solution to this problem will need to be correspondingly complex and multifaceted. \par
This study analyzes many variables of the gun control debate. It reviews past investigations and fills in the blanks. It then proceeds to detail the implications of the results. The study finds that high levels of criminal activity lead to more firearm death. It also finds that the percentage of African Americans residing in a state and the unemployment rate positively relate to gun-related deaths. Kwon is Emeritus Professor at the Saint Louis University School of Business. He is an expert in multivariate statistical analysis, and employs this technique to confront the holistic measure of gun control legislation. The largest shortcoming of this study is Kwon's lack of expertise regarding gun control. This fact is also this study's golden ticket. Kwon enters the debate with eyes wide shut, and takes the data as it is. This frank analysis of the efficacy of gun legislation directly addresses the gun control policy topic.
\bibendote

\bibent Loftin, Colin, et al. “Mandatory Sentencing and Firearms Violence: Evaluating an Alternative to Gun Control.” Law \& Society Review, vol. 17, no. 2, 1983, pp. 287–318. JSTOR, https://doi.org/10.2307/3053349. Accessed 1 Jan. 2023.
\bibannote
Abstract: \par Michigan's Felony Firearm Statute (Gun Law) imposed a two-year mandatory add-on sentence for defendants convicted of possession of a firearm in the commission of a felony. The Law was widely advertised with proponents claiming that it would introduce greater equity in sentences, ensure certainty of punishment, and decrease violent crime in the state. We examine the processing of these Gun Law cases in Detroit Recorders Court, as well as the effects of the law on crime, and find that most of the goals of the Law's proponents are not met. Notwithstanding a rigid prosecutorial policy which prohibited plea bargaining in these gun cases, alternative mechanisms developed to mitigate the Law's effects and, in most instances, to preserve the "going rate" for various crime categories. Similarly, using an interrupted time-series model, we are unable to uncover effects of the law, or the associated publicity campaign, on violent crime. \par
In this paper, the authors seek to show how Michigan's Felony Firearm Statute has not met the goals which it's proponents had put forth. The data is striking, in an example of the failure of gun legislation. Colin Loftin is co-director of the Violence Research Group, a research collaboration with colleagues at the University at Albany and the University of Maryland, that conducts research on the causes and consequences of interpersonal violence. This paper approaches this subject in an attempt to disprove the laws' proponents, but shows the benefits of these legal proceedings in the process. This paper shows us how gun legislation can miserably fail.
\bibendote

\bibent Seitz, Steven Thomas. “Firearms, Homicides, and Gun Control Effectiveness.” Law \& Society Review, vol. 6, no. 4, 1972, pp. 595–613. JSTOR, \url{https://doi.org/10.2307/3052950}. Accessed 1 Jan. 2023.
\bibannote
This paper examines the effects of gun control legislation. Presented is data to suggest that homicides directly correlate to firearm homicides, thus ruling out that most deadly attacks are simply a result of a single-minded intention to kill. Steven T. Seitz is a Political Science Educator at the University of Minnesota. This paper is based on a testimony he delivered to the Judiciary Committee of the Minnesota State Senate in 1971. Seitz relies heavily on research conducted by other prominent names, and provides little development in the data and analysis of such data, as do others, such as Zimring (one refered to in this paper). This paper does, however, develop the presentation of the impacts of gun control, allowing the public to better understand the ramifications of gun legislation.
\bibendote

\bibent Smith, Tom W. “Public Opinion about Gun Policies.” The Future of Children, vol. 12, no. 2, 2002, pp. 155–63. JSTOR, \url{https://doi.org/10.2307/1602745}. Accessed 1 Jan. 2023.
\bibannote
Smith brings an update to the public polling on gun control measures. For over 40 years, attitudes remain stable. Americans strongly support measures to regulate firearms, short of outright prohibition. Tom W. Smith (Ph.D. University of Chicago) is the Senior Fellow and Director, National Opinion Research Center (NORC) at the University of Chicago, Center for the Study of Politics and Society. While Smith sees a trend in Gun control sentiment, this seems to be largely based upon his sample set and the radical differences from one survey option to another. This paper shows that there is substantial evidence that more restrictions on firearms would be in the public's interest. Therefore making gun control measures effective. 
\bibendote

\bibent Stolzenberg, Lisa, and Stewart J. D’Alessio. “Gun Availability and Violent Crime: New Evidence from the National Incident-Based Reporting System.” Social Forces, vol. 78, no. 4, 2000, pp. 1461–82. JSTOR, \url{https://doi.org/10.2307/3006181}. Accessed 1 Jan. 2023.
\bibannote
Abstract: \par Using four years of county-level data drawn from the National Incident-Based Reporting System (NIBRS) for South Carolina and a pooled cross-sectional time-series research design, we investigate whether gun availability is related to violent crime, gun crime, juvenile gun crime, and violent crimes committed with a knife. We contribute to the literature by distinguishing between illegal and legal gun availability and by using a comprehensive measure of gun crime. Results show a strong positive relationship between illegal gun availability and violent crime, gun crime, and juvenile gun crime. Little or no effect for the legitimate gun availability measure is observed in any of the estimated models. Findings also reveal that illegal guns have little influence on violent crimes committed with a knife. Offenders seem not to be substituting knives or other cutting instruments when illegal firearms become less available. A supplemental analysis also indicates no evidence of simultaneity between gun availability and violent crime. The strong and consistent effect of illegal rather than legal gun availability on violent crime has important policy implications, because it suggests that greater attention should be directed at devising ways for legitimate gun owners to better secure their weapons. \par
This paper analyzes whether gun availability is related to various violent crimes. It is interesting to see that when guns are not available, criminals opt to use a knife or alternative cutting instrument. Lisa Stolzenberg is a Professor and Chair of Criminology and Criminal Justice for the Steven J. Green School of International \& Public Affairs at Florida International University. This study uses information for a small locale (South Carolina). Regardless, this study is helpful in understanding what effects gun control have upon violent crime rates (likely none).
\bibendote

\bibent WOLF, CAROLYN REINACH, and JAMIE A. ROSEN. “MISSING THE MARK: GUN CONTROL IS NOT THE CURE FOR WHAT AILS THE U.S. MENTAL HEALTH SYSTEM.” The Journal of Criminal Law and Criminology (1973-), vol. 104, no. 4, 2014, pp. 851–78. JSTOR, \url{http://www.jstor.org/stable/44113411}. Accessed 1 Jan. 2023.
\bibannote
Abstract: \par Recent gun control legislation aimed at removing guns from the hands of the mentally ill in order to reduce violence is misguided. In fact, this only contributes to the mistaken belief that there is a direct link between mental illness and violence. This Article suggests that instead, policymakers should be focusing on modifying existing restrictive mental health laws and increasing the funding needed to provide adequate mental health services in the community. Family members, the community, and the individuals themselves must have access to adequate resources and support systems to increase the individuar's chance of recovery and stability. In light of recent tragedies, a better solution to reducing gun violence includes offering community programs and preventive training in educational and workplace environments to allow for early detection and intervention. The current system does not support those in need of treatment and only serves to exacerbate the stigma associated with mental illness. \par
This paper flips the gun control debate on its head. In the wake of gun legislation, Wolf addresses the very real reason that mass shootings occur: The Mental Health Crisis in America. This paper considers the barriers to seeking treatment, and finds that the path of least resistance is a public display of force. Carolyn Reinach Wolf is an Executive Partner in the law firm of Abrams Fensterman. She is an expert on mental health. Her practice focuses on a family-oriented approach to mental healthcare. This article narrowly approaches the gun control debate. It leaves gang violence and other similar violent crime out of the picture. This paper does, however grant a unique perspective as to why gun control may not be the answer, and therefore not effective. 
\bibendote

\bibent Wolpert, Robin M., and James G. Gimpel. “Self-Interest, Symbolic Politics, and Public Attitudes toward Gun Control.” Political Behavior, vol. 20, no. 3, 1998, pp. 241–62. JSTOR, \url{http://www.jstor.org/stable/586530}. Accessed 1 Jan. 2023.
\bibannote
Abstract: \par Numerous studies have found that immediate and tangible self-interest has a minimal influence on public attitudes toward many policy issues. We examine public attitudes toward gun control in order to determine whether gun owners exhibit distinctive policy preferences. Our results indicate that self-interest strongly influences public preferences on gun control and that banning handguns evokes stronger self-interest effects than banning assault weapons or imposing a waiting period on purchases of firearms. We conclude by discussing why gun control evokes self-interested calculations among gun owners, the implications of our findings for self-interest theory, and suggestions for further lines of research. \par
Wolpert takes the time to examine the public perception of the gun control stances. The highlight of this paper is the polling of respondents who make their opinion known, whether they favor handgun bans, assault weapons bans or waiting periods. Robin M. Wolpert is Professor Emeritus of Statistical Science at the Duke Nicholas School of the Environment. Originally trained as a mathmetician, Wolpert is now a stochastical modeler. Usually this involves predicting things that haven't been measured. This paper helps us determine whether gun control efforts would satiate the appetite of the public, at large. 
\bibendote

\bibent Zimring, Frank. “Is Gun Control Likely to Reduce Violent Killings?” The University of Chicago Law Review, vol. 35, no. 4, 1968, pp. 721–37. JSTOR, \url{https://doi.org/10.2307/1598883}. Accessed 1 Jan. 2023.
\bibannote
By analyzing data from the Police Department of the City of Chicago on reported criminal homicides and serious criminal assaults, this source attempts to settle the gun control debate. Zimring found that many homicides are related to variable states of intention, rather than the assumption that criminal homicide is driven by somebody's destructive goal. Frank Zimring is the Professor of Law atthe UC Berkeley School of Law. He is only, however, able to take into account data from the City of Chicago. The limited nature of such data is not indicative of the data for the entire nation. This work directly addresses the effects of Gun Control and the genus of deadly attack.
\bibendote

\pagebreak
\section{What Has Been Learned}
During this process, much has been learned about the effects and efficacy of Gun Control. While many opinions were presented in my searches, there were also many data-driven articles which helped to show the facts of what does and doesn't work. This all, presented in light of the acknowledgement of the legal constraints, comes together to show how the public could never be satisfied and the debate will continue while nothing, much, changes. 
\par
Democrats are generally in support of Gun Control. They desire strict restrictions, and generally lean towards the side of the debate which would argue that the second ammendment ought to be repealed, and that all guns should be bought back so that no one has them. 
\par
Republicans are generally opposed to stricter restrictions on gun availability. They hold that the second amendment should be interpreted as a constitutional restriction on the government, that the populace should not be restricted from owning or bearing firearms for any reason. 
\par
Libertarians are generally opposed to Gun Control, often leaning towards an anarchist perspective. They structure their policy around mottos such as "Don't tread on me", suggesting that they should be allowed to do what they want and others should be allowed to do what they want, also. 
\par
The majority of America does not seem to agree on what should be done, regarding Gun Control. Often polls hover at about 50\% for more gun restrictions, and 50\% against. Firearm owners find the subject more salient, and are therefore more often involved, actively, by spending or sending money, voting, protesting, or even getting involved in public altercations. 
\par
I do not believe that America needs to restrict gun ownership any more than it has. Rights regarding gun legislation should be returned to the states. If Texas wants people to have automatic firearms, they should be allowed them. If California wants to restrict people to only carrying muskets, they should be allowed to. The constitution has been interpreted to disproportionately prefer the powerful and political citizens of this country. 
\par
I was motivated to choose this topic, because, having traveled much of the world and much of the United States, I have seen many cultures with different opinions on how guns ought to be regarded. I enjoy having a firearm and being capable to match a threat in my daily life. The restriction on such an action is imposition of restriction on an inalienable right. 
\end{document}
