
\documentclass[12pt,a4paper,english]{article}
\usepackage{helvet}         % set font to helvetica
\renewcommand{\familydefault}{\sfdefault}
\usepackage{sectsty}        % allow redefinition of section command formatting
    \allsectionsfont{\normalsize}
\usepackage[centering,noheadfoot,margin=1in]{geometry}
\usepackage{lipsum}
\setlength{\parindent}{0em}
\usepackage{hanging}

% https://tex.stackexchange.com/questions/30720/footnote-without-a-marker
\newcommand\blindfootnote[1]{
  \begingroup
  \renewcommand\thefootnote{}\footnote{#1}
  \addtocounter{footnote}{-1}
  \endgroup
}
\AtEndDocument{\blindfootnote{\textrm{This document proudly made using \LaTeX{}}}}

\begin{document}
\section*{\centering{Audience Analysis of Rhetorical Decisions}}
\begin{center}
Dean Gadberry
\end{center}
This report intends to examine the rhetorical decisions made by authors as they design documents intended for different audiences.
\section*{Expert Audience: Agriculture Literature Review}
\textbf{Document Design.}

\par
\textbf{Style and Tone.}

\textbf{Source Usage.}

\section*{Non-Specialist Audience: "How to Plant a Three Sisters Garden" Article}
\textbf{Document Design.}
\par
\textbf{Style and Tone.}

\textbf{Source Usage.}

\section*{\centering{Works Cited}}
\begin{hangparas}{2em}{1}
  Boeckmann, Catherine. “The Three Sisters: Corn, Beans, and Squash.” Almanac.com, The Old Farmer's Almanac, 26 May 2022, https://www.almanac.com/content/three-sisters-corn-bean-and-squash. 
  \par
  Ngapo, Tania M., et al. “Historical Indigenous Food Preparation Using Produce of the Three Sisters Intercropping System.” Foods, vol. 10, no. 3, Mar. 2021, p. 524. Crossref, https://doi.org/10.3390/foods10030524.
\end{hangparas}
\end{document}
